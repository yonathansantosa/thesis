\chapter{Conclusion}
\label{chap:conc}

Given some training examples of pretrained word embedding, machine
learning can be used to generate embedding for
\textit{out-of-vocabulary} embedding which does not exist in the
pretrained word embedding vocabulary collection. This method however,
needs time and resources to be trained to generate a good
approximation of the \textit{out-of-vocabulary} embedding.

\section{OOV Handling Model against Randomly Initialized OOV Embedding}
For \textit{out-of-vocabulary} embedding to be used in the downstream
tasks, one can assign randomly generated embedding for each
\textit{out-of-vocabulary} or by assigning single token with one
vector representation replacing all \textit{out-of-vocabulary}
entries. This approach has its advantages such as cheaper computation
to generate embedding and acceptable results for POS-tagging tasks,
the training on the downstream tasks might need more time or more
sophisticated architecture to be able to keep up with the one that
uses OOV handling model such as \textsc{Mimick} or the proposed model
from this research. As evidence, with similar setup as the OOV
handling model the randomly generated embedding for
\textit{out-of-vocabulary} has lower accuracy for POS-tagging tasks.
The randomly generated embedding for \textit{out-of-vocabulary}
entries method could produce POS-tagging results for three different
pretrained embedding averaged around 90\% and by using OOV handling
model the accuracy increased to be around 93\%.

\section{CNN against LSTM for Extracting Text Features}
As shown by the results of the downstream tasks, CNN model generally
outperforms LSTM implemented in \textsc{Mimick}. As aforementioned
before, \textsc{Mimick} uses the last state of the hidden state from
the bi-LSTM architecture. This renders the intermediate sequence that
could be important for predicting the embedding lost when the cell
gate decided to drop previous information. By using CNN to extract
text features, the intermediate sequence that appears somewhere in the
middle of a word can still be inferred. With some generalization, the
CNN features also able to use one kernel for several sequence that has
certain similarities in terms of sequence of embeddings hence the
higher accuracy and Spearman's rank correlation values.

\subsection{POS-tagging}
In POS-tagging tasks, CNN model accuracy is around 1\% higher than
\textsc{Mimick}. On top of the CNN architecture, some batch
normalization were added to make the training converge faster. The
only disadvantages of the CNN model is that it has more parameters
to learn, meaning higher complexity and slower training time. The
advantages of the CNN model is that it can be implemented in a
parallel computing systems while LSTM needs the previous time-steps to
calculate current time-steps.

\subsection{Word Similarity Task}
Word similarity task were used to evaluate the distance between two
words according to human subject. In this task, the word generated
embedding expected to have similar degree of similarities between two
embeddings from two words. In this task, the CNN model were able to
achieve higher $\rho$ value than \textsc{Mimick}.

\section{Future Works}
Even both model \textsc{Mimick} were able to produce embeddings for
\textit{out-of-vocabulary} words, both model were not really able to
define a function that could generate word embedding mimicking
existing pretrained embedding since updating pretrained embedding with
new dataset will requires retraining the embedding. Instead of
retraining the embedding, finding function or model that are able to
mimick the pretrained embedding will save computation time at least
until the language itself is changing into something completely
different making the word embedding becomes obsolete.