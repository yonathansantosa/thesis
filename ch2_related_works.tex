\chapter{Literature Review}
\label{chap:literature}

% This is the introduction to the thesis.\footnote{And this is a
% footnote.}  The conclusion is in Chapter on page
Word2vec is one of word embedding that is trained using skip-gram
model \citep{Distributed2013mikolov}. A word $w(t)$ used as an input
and its context word, for example context word with windows of 4 are
$w(t-2), w(t-1), w(t+1),$ and $w(t+2)$, used as the target and the
projection from input to the output is used as the representation of
the input $w(t)$ that is usefull to predict the context words. This
model is highly dependant on the corpus completeness. More examples
and vocabulary a corpus has the better the representation of the
embeddings since more information will be able to be learned. Word2vec
model has no oov handling, meaning either random vector or unknown
\textit{\textless UNK\textgreater} embedding will be used.

Polyglot embedding ........

Dict2vec is yet another embedding that is trained by looking up
definitions of words from cambridge dictionary
\cite{tissier2017dict2vec}. This embedding was created because the
previous method is trained with unsupervised manner, meaning that
there is no supervision between pairs of words. There might exists
pair of words that are actually related but do not appear enough
inside a corpus making it harder for the model to find connection.
Thus, this model is trained by creating sets of strong and weak pairs of
words, then move both pairs closer and further respectively based on
the pairs. The model then evaluated using several word similarity
tasks to show imporvements over vanilla implementation of word2vec and
fasttext.

Part-of-Speech-Tagging (postagging) is a process of determining
grammatical category (tag) of given word in a certain sentence. In English,
exist words that has ambiguous grammatical category, such as word
"tag" can be either noun or verb depends on the usage of it
\citep{apractical1992cutting}. To tackle this problems, many
researchers proposed to use mathematical models or statistical models
namely hidden markov model \citep{apractical1992cutting}, n-grams
\citep{tnt2000Brants}, and neural network model
\citep{finding2015ling}. In this research, the neural network model
will be implemented to serve as the downstream task. This model is a
recurrent neural network (RNN) that took sequence of word embeddings
representing a sentence or parts of sentence then categorize each word
embedding for its tag.

As aforementioned above, some word embedding model such as Word2vec
has no oov handling, thus creating a model to predict such word
becomes research interest. One of the model that tries to tackle this
problem successfully used bi-LSTM to predict embedding given sequence
of characters from a word from a pretrained embedding
\citep{mimicking2017Pinter}. As a result, oov embeddings are able to
be predicted without the needs of knowing lexicon or model used for
creating the word embedding. The results then tested to do downstream
task namely postagging.

N-grams often used to capture word features. N-grams relies on
characters that make up a word, later on a sentence. With character
embedding and convolution neural network (CNN), n-grams can be calculated by
convoluting sets of characters embedding with the kernel size of $n
\times d$ for $n$ is the number of grams and $d$ is the dimension of
the embeddings. This CNN n-grams then can be used to create a neural
language model \citep{character2015kim}.