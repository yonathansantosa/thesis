\chapter{Implementation}
\label{chap:implementation}

\section{Preparation}
    \subsection{Dataset Preparation}
        \subsubsection{Character Embedding}
            The character dictionary consists of first 128 ASCII
            characters with deletion for non character types (the
            first 32 ASCII symbols). Then the character embedding was
            initialized by randomly generated from normal distribution
            with $\mu = 0$ and $\sigma = 1$. Another entries such as
            unknown token ''\textit{\textless unk\textgreater}'', starting
            token ''\textit{\textless s\textgreater}'', ending token
            ''\textit{\textless e\textgreater}'' and padding token
            ''\textit{\textless p\textgreater}'' were added along with the
            random character embedding initialization.

        \subsubsection{Pre-trained Word Embedding}
            In order to train the model, pre-trained embedding was
            needed since the model would tries to predict embedding
            from a known vocabulary $v_i \in \mathcal{V}$ with its known
            embedding $e_i \in \mathcal{W}$. For this purpose, word2vec
            which trained on Google news dataset using skip-gram model
            containing 3 million words and phrases (name, hyperlink,
            connected words, etc.) was used
            \citep{Distributed2013mikolov}. The embedding contains
            300-dimensional vectors. Word2vec \footnote{Pretrained
            embedding available at 
            \url{https://code.google.com/archive/p/word2vec/}}
            contains words that frequently appears as a phrase, for
            instance the word New and Jersey appears frequently side
            by side because both of this words form name of a state in
            United States of America. In the original vocabulary, this
            phrase might be written as "New\_Jersey", thus for the
            purpose of simplifying the input and the downstream tasks,
            this kind of phrases was not included. On top of phrases,
            the original vocabulary also includes hyperlinks which
            usually contains "http". Such entries will also be
            removed. Only first 40 thousands words with removal if the
            word contains "\_" (underscore) or "http" was used in this
            research.

            Another pre-trained embedding is
            polyglot\footnote{Pretrained embedding available at
            \url{https://polyglot.readthedocs.io/en/latest/index.html}}
             which contains multilingual embeddings
            \citep{polyglot2013alrfou}. For this research, only
            English embedding that contains around 100.000 words with
            60-dimensional vector representations will be used. This
            pre-trained embedding was also used in OOV handling model
            \textsc{Mimick} which used as baseline model
            \citep{mimicking2017Pinter}.

            Dict2vec\footnote{Pretrained embedding available at
            \url{https://github.com/tca19/dict2vec}} is yet another
            word embedding that was trained based on the definition of
            a word in dictionary \citep{dict2vect2017tissier}. Each
            words location in Euclidean space was determined by the
            appearance of another words in the definition that is
            defined by the Cambridge dictionary. Originally, this
            embedding was tested using word similarity tasks with
            removal for OOV words.

        \subsubsection{Word Similarity Dataset}
            Several word similarity datasets were used in order to
            increase the pair examples since for the datasets that are
            collected, the highest number of pairs is just above 3000
            pairs. The datasets used for word similarity tasks are
            Card-660 \citep{card660:pilehvar-etal:2018}, MC-30
            \citep{mc30:strongContextualHypothesis}, MEN-TR-3k
            \citep{mentr3k:bruni-etal-2012-distributional}, MTurk-287
            \citep{mturk287:Radinsky:2011:WTC:1963405.1963455},
            Mturk-771
            \citep{mturk771:Halawi:2012:LLW:2339530.2339751}, RG-65
            \citep{rg65:Rubenstein:1965:CCS:365628.365657},
            RW-STANFORD \citep{rw:luong-etal-2013-better}, SimLex-999
            \citep{simlex999:hill2014}, YP130
            \citep{yp130:inproceedings}, VERB143
            \citep{vp143:baker-etal-2014-unsupervised}, and Wordsim353
            \citep{wordsim353:2002:PSC:503104.503110}.

    \subsection{Programming Language and Tools}
        The model was trained using PyTorch 1.1.0 machine learning
        library on top of Python 3.6 \citep{pytorch2017paszke}. Most
        of the basic functions, for instance 2d convolution layer, 2d
        maxpool layer, fully connected layer, bi-LSTM, many
        activation functions and loss functions, and other mathematical
        functions were already implemented as a library in PyTorch,
        thus will serve enough for the purpose of this research.

    \subsection{Hardware}
        The model was trained on the freely available Google
        Colaboratory\footnote{Google Colaboratory available at
        \url{https://colab.research.google.com/}} which gives
        randomized hardware specification based on the availability,
        thus the exact hardware configuration used cannot be
        determined. The GPU engine was used to train the model.
        Nevertheless, this only affects the time needed to train the
        model and not the results.

\section{Training}
    \subsection{Training OOV model}
        Firstly, the pretrained embeddings acted as the datasets were
        shuffled and split up into train-val set with $80\%$ and
        $20\%$ quota respectively with minibatch size of 64. The word
        $w_i \in \mathcal{V}$ becomes the input of the model and the
        word embedding $e_i \in \mathcal{W}$ becomes the target. The
        input word split into sequence of characters and starting
        token ''\textit{\textless s\textgreater}'' and ending token
        ''\textit{\textless e\textgreater}'' were added at the beginning
        and at the end of the word respectively. For every minibatch,
        the longest word will be used as the maximum length. Every
        word that was shorter than the longest word will be padded
        with a padding token ''\textit{\textless p\textgreater}''.
        % For the proposed model, padding token was added at the
        % beginning and at the end of the word to make single
        % character entries at the beginning and at the end of the
        % word to be able to be processed by the CNN. 
        The sequence of characters then transformed into character
        embeddings then processed by the model producing the predicted
        word embedding. The \textsc{Mimick} model was trained with
        learning rate $\eta = 0.01$ for polyglot and $\eta = 0.1$ for the
        rest with no dropout as dropout does not work well just as in
        the original implementation of \textsc{Mimick}
        \citep{mimicking2017Pinter}. On the other hand, CNN model was
        performing better with dropout when the model was pre-tested
        using different parameters. In summary, the hyperparameters
        setting is shown in table \ref{tab:hyperparameter}.

        \begin{table}[]
            \centering
            \caption{OOV Handling Model Parameters}
            \label{tab:hyperparameter}
            \begin{tabular}{@{}lcc@{}}
                \toprule
                \textbf{Hyperparameter} & \multicolumn{1}{c}{\textbf{\textsc{Mimick}}} & \multicolumn{1}{c}{\textbf{CNN}} \\ \midrule
                Train-Val split & \multicolumn{2}{c}{$80\%$; $20\%$}\\
                Batch size & \multicolumn{2}{c}{64} \\
                Epoch & \multicolumn{2}{c}{100} \\
                Momentum & \multicolumn{2}{c}{0.5} \\
                Learning Rate ($\eta$) & [0.01; 0.1] & 0.1 \\
                Dropout & 0 & 0.5 \\
                Num features & [50; 100; 200; 300] & [20; 50; 100; 150] \\ \bottomrule
            \end{tabular}
        \end{table}

        For the proposed model, the character embeddings were
        processed with CNN n-grams following
        \cite{convolutional2014kim} architecture for word n-grams.
        N-grams with window size $n = [2, 3, 4, 5, 6, 7]$ were used
        with respective kernel size $n \times b$ to simulate n-grams.
        Different sets of n-grams sizes, for instance $n = [2, 3, 4]$
        or $n = [5, 6, 7]$ can be used instead of the whole n-grams
        sizes, but it has worse results in the pre-testing phase. Thus
        those settings were not used. The results on the downstream
        tasks for the different settings would compared with the whole
        model and with the \textsc{Mimick} model.

        After max-over-time pooled, the vector representation was
        passed into two layers of fully connected network with batch
        normalization layer for each layer to produce the predicted
        word embedding. Batch normalization helps with the training
        process by making all of the minibatch data to have similar
        distribution for each layer
        \citep{batchnorm:DBLP:journals/corr/IoffeS15}. After the
        embedding was predicted, the error was calculated using Mean
        Squared Error as mentioned on equation \ref{eq:errorf} and the
        derivative was calculated then back-propagated to update the
        model parameters.

        Similar datasets were used to train \textsc{Mimick} with
        parameters taken from the original paper
        \citep{mimicking2017Pinter}. To find the optimal parameters
        such as learning rate, number of features/hidden neurons, and
        number of epoch, several number of tests with different number
        of parameters needed to be done as preliminaries. The summary
        of the hyperparameters that are picked after the preliminary
        tests are shown in table \ref{tab:hyperparameter}.

        % \begin{table}[]
        %     \centering
        %     \caption{OOV Handling Model Parameters}

        %     \begin{tabular}{@{}lcl@{}}
        %         \toprule
        %         \textbf{Hyperparameter} & \multicolumn{1}{l}{\textbf{\textsc{Mimick}}} & \textbf{CNN} \\ \midrule
        %         Learning Rate ($\eta$) & \multicolumn{2}{c}{0.1} \\
        %         Batch size & \multicolumn{2}{c}{64} \\
        %         Epoch (word2vec) & \multicolumn{2}{c}{1000} \\
        %         Epoch (polyglot-en) & \multicolumn{2}{c}{100} \\
        %         Epoch (dict2vec) & \multicolumn{2}{c}{1000} \\
        %         Momentum & \multicolumn{2}{c}{0.5} \\
        %         Num features & [100; 700] & \multicolumn{1}{c}{100} \\ \bottomrule
        %     \end{tabular}
        % \end{table}
    
\section{Evaluating with Downstream Tasks}
    \subsection{Part-of-Speech Tagging}
        For POS-tagging task, the readily brown corpus and its tagset
        from NLTK\footnote{Available at \url{https://www.nltk.org/}}
        were used to test the performance of the model as well as the
        previous \textit{state-of-the-art}. In this task, two kinds of
        evaluation methods were done and will be explained below.
        Before going into the method, there are several steps to
        prepare the datasets for training the POS-tagger. Firstly, the
        POS-tag dataset words were collected as sets $\mathcal{S}$.
        This was done in order to collect the vocabulary uniquely
        since in Python set elements are unique. Secondly, each
        elements in the set $\mathcal{S}$ was scanned whether it is
        in-vocabulary or out-of-vocabulary. If the element turns out
        to be OOV, this element which is a word, will be added into
        the vocabulary $\mathcal{V}$ and its embedding, either a zero
        vector, a randomly initiated value, or a generated embedding
        from the OOV model depending by the method used for testing,
        will be added to the original embedding $\mathcal{W}$. The
        dataset were split up into 80\% for train set and 20\% for
        test set.

        The first method was to train the POS-tagger by using a
        pre-trained embedding as the baseline result. The original
        pre-trained embedding entries $\mathcal{W}$ were concatenated
        with the OOV model predicted embeddings $f(\text{OOV})$ so
        that each word $v \in \mathcal{V}$ has vector representation.
        In short, the new embedding $\mathcal{W}_{new} = \mathcal{W}
        \cup f(\text{OOV})$. The new pre-trained embedding
        $\mathcal{W}_{new}$ will not be trained in this method by
        setting the parameters to be fixed values.

        The second method was to train the OOV model together with the
        POS-tagger. For this method, for each OOV entries added into
        the vocabulary $\mathcal{V}$, a zero vector representing the
        embedding of the OOV would be added to the embedding
        $\mathcal{W}_{new}$ that later on replaced with the generated
        embedding $f(\text{OOV})$ if the input word is OOV as
        explained in the equation \ref{eq:embeddingmask} by masking
        the entries. Thus the OOV model would still be able to be
        trained if the input turns out to be OOV while it is
        non-trainable if it is in-vocabulary input. This method was
        used by both models then the results were compared.

        Lastly, it is similar to the second method but the pre-trained
        embedding was set to be a trainable parameters. Using similar
        method of masking, the gradient would flow separately depending
        on the embedding used, whether it is from the pre-trained
        embedding or from the embedding generated by the OOV model.

        On top of different methods used in this task, the sentence
        length was limited to 5 words. If the sentence length was less
        than 5 words then padding tokens were added at the end of the
        sentence to make the length becomes 5 words and if the
        sentence's length was more than 5 words, the slice of the
        sentence were randomly selected. Note that this padding token
        is different than the padding token of the character
        embedding. The padding token ''\textit{\textless
        p\textgreater}'' is a zero vector with dimension similar with
        the pre-trained embedding $\mathcal{W}$. The POS-tagger were
        trained with 100 epoch with 6 different seeds for the random
        number generation. The learning rate was $0.1$ with dropout
        for CNN model $0.5$ and no dropout for \textsc{Mimick} since
        it makes the model performs worse based on preliminary tests.
        The momentum for the training was $0.5$. In summary the used
        parameters are shown in table \ref{tab:hyperparameterpostag}.

        \begin{table}[]
            \centering
            \caption{POS-Tagger Model Parameters}
            \label{tab:hyperparameterpostag}
            \begin{tabular}{@{}lcc@{}}
                \toprule
                \textbf{Hyperparameter} & \multicolumn{1}{c}{\textbf{\textsc{Mimick}}} & \multicolumn{1}{c}{\textbf{CNN}} \\ \midrule
                Learning Rate ($\eta$) & \multicolumn{2}{c}{0.1} \\
                Batch size & \multicolumn{2}{c}{64} \\
                Epoch & \multicolumn{2}{c}{100} \\
                Momentum & \multicolumn{2}{c}{0.5} \\
                Training seed & \multicolumn{2}{c}{[64; 20; 128; 0; 5;
                10]} \\
                Dropout & 0 & 0.5 \\ 
                \bottomrule
            \end{tabular}
        \end{table}

    \subsection{Word Similarity}
        Word similarity task is quite straight forward. Given pairs of
        words, if the word was an OOV in the pre-trained embedding, the
        word embeddings were predicted from OOV models then the cosine
        similarity of the word embeddings were calculated and compared
        between the two models, else the original embedding from the
        pre-trained embedding was used. 

        The baseline embeddings used in this task, dict2vec
        \citep{dict2vect2017tissier}, was also tested using random OOV
        embedding by randomly giving OOV word random embedding and
        choosing the maximum results from five tries for each dataset.
        On top of that, other two pre-trained embedding used in this
        research were also used for comparing the results between the
        two models.